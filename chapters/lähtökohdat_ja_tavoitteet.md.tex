\hypertarget{luxe4htuxf6kohdat-ja-tavoitteet}{%
\chapter{Lähtökohdat ja
tavoitteet}\label{luxe4htuxf6kohdat-ja-tavoitteet}}

\hypertarget{nordhealth-oy-lyhyesti}{%
\section{Nordhealth Oy lyhyesti}\label{nordhealth-oy-lyhyesti}}

Nordhealth Oy on vuonna 2001 perustettu ohjelmistopalveluyritys, joka
tekee toiminnanohjausjärjestelmiä kahdelle eri toimialalle: Provet
Cloud\cite{ProvetCloudHomepage} -järjestelmää eläinlääkäriklinikoille ja
Diarium-järjestelmää\cite{DiariumHomepage} terapeuteille. Molemmat
järjestelmät ovat web-pohjaisia sovelluksia. Niitä siis käytetään
web-selaimen kautta.

Nordhealthia voisi luonnehtia tyypilliseksi ohjelmistoalan yritykseksi:
se tekee erityisalalle suunnattua toiminnanohjausjärjestelmää.
Tyypillinen on myös järjestelmien kehityskaari: alunperin ne ovat olleet
työpöytäsovelluksia, ja siitä Nordhealth on muuntanut ne LAMP\footnote{Linux,
  Apache, MySQL, PHP}-alustalla toimiviksi web-sovelluksiksi. Kun Web on
kehittynyt, on otettu suunnaksi sovellusten siirtäminen julkiseen
pilveen, ja käyttöliittymän rakentaminen erilliseksi yhden sivun
JavaScript-sovellukseksi.

Diarium on Nordhealthin kahdesta järjestelmästä vanhempi. Se on
suunnattu terapeuteille: fysio-, toiminta-, puhe- ja psykoterapeuteille.
Järjestelmä on laajentunut yksinkertaisesta
potilaskortistojärjestelmästä suurenkin terapia-alan yrityksen tarpeita
vastaavaksi toiminnanohjausjärjestelmäksi, ja se on lisäksi myös
Valviran tarkoittama A-luokan potilastietojärjestelmä. Diariumia
käyttävä terapeutti voi siis lähettää käyntikirjaukset potilastiedon
sähköiseen Kanta-rekisteriin.

\hypertarget{tarve-tyuxf6lle}{%
\section{Tarve työlle}\label{tarve-tyuxf6lle}}

Diariumin ikä näkyy tietomallin monimutkaisuutena. Vuosien saatossa
tehty kehitystyö on tehnyt ohjelmiston joistain osista hankalia
ymmärtää. Ohjelman kehittäminen on myös aloitettu aikana, jolloin
sovelluksia ei automaattisesti tehty yhden sivun sovelluksiksi.

Nykyään tämä kuitenkin on normi, ja ohjelmaan on jo vuosia kehitetty
REST-rajapintaan nojaavia toiminnallisuuksia. Samalla kehitystyön myötä
on huomattu, että REST soveltuu huonosti joihinkin monimutkaisen
toiminnanohjausjärjestelmän vaatimiin tehtäviin.

Olisiko REST-rajapinnalle sopivampi vaihtoehto? Entä voisiko rajapintaa
rakentaessa myös parantaa ohjelmiston sisäistä tietomallia? Tämä
säästäisi aikaa ja vaivaa, ja nopeuttaisi ohjelmiston jatkokehitystä.

\hypertarget{insinuxf6uxf6rityuxf6n-tavoite}{%
\section{Insinöörityön tavoite}\label{insinuxf6uxf6rityuxf6n-tavoite}}

Tämän insinöörityön tavoitteena on selvittää, onko ohjelmiston
tietomallia mahdollista parannella rakentamalla GraphQL-rajapinta.
Lisäksi työn myötä on tavoitteena löytää parempi tietomalli osaan
Diarium-sovellusta.

Kolmas tärkeä tavoite on kehittää työmenetelmä, jonka avulla tietomallia
on mahdollista korjata. Näin työssä tehtyjä havaintoja on mahdollista
hyödyntää jatkossa.

Projektin edetessä rakennan pienen GraphQL-rajapinnan ja sitä
hyödyntävän prototyyppisovelluksen. Tämä toimii kokeilukenttänä, jonka
kautta työmenetelmää ja tietomallia etsitään.
