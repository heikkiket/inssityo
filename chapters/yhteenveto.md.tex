\hypertarget{yhteenveto}{%
\chapter{Yhteenveto}\label{yhteenveto}}

Tässä insinöörityössä pyrittiin kohentamaan ohjelmiston tietomallia
kehittämällä GraphQL-rajapinta
\glsdisp{ddd}{sovellusaluevetoisen suunnittelun} keinoin. Tavoitteena
oli parantaa nykyistä tietomallia ja luoda työprosessi tietomallin
parantelemiseen. Samalla etsittiin vastausta kysymykseen, miten hyvin
GraphQL-rajapinta teknologiana sopii tällaiseen prosessiin.

Saadakseni vastauksia kysymyksiin laadin pienen prototyyppisovelluksen,
jonka tehtäväksi asetimme yhdessä tilaajan edustajien kanssa
laskutukseen liittyvän konkreettisen ongelman ratkaisemisen. Keskeiset
vastaukset työn nostamiin kysymyksiin tulivat juuri tämän sovelluksen
kehitysprosessin kautta. Samalla kehitystyö muovasi lopullista
työprosessia.

Tehdyn kokeilun perusteella GraphQL-rajapinta soveltuu hyvin
sovellusaluevetoisen suunnittelun tarpeisiin. Tämä johtuu sen
verkkomaisesta luonteesta, jolla on helppo mallintaa sovellusalueen
käsitteiden keskinäisiä suhteita. Se, että GraphQL-verkko on nimenomaan
rajapinta, helpottaa käsitteellisen mallin erottamista omaksi
kokonaisuudekseen sovelluksen sisällä, ja irrottaa mallin
konkreettisesta teknologiasta.

Koska GraphQL-rajapinta määritellään \glsdisp{dsl}{täsmäkielen} avulla,
mallia on helppo muokata osana iteratiivista kehitysprosessia. Tämä
mahdollistaa tutkimusmatkat ja kokeilut erilaisilla malleilla, kun
käsitteiden välisiä suhteita voidaan muuttaa ensiksi skeemassa, ja vasta
sitten taustalla olevassa koodissa.

Projektin aikana löysin kohteeksi valitun laskutuksen ongelman
ratkaisevan tietomallin, ja luonnostelin \emph{notkean tietomallin
parantelun} periaatteet. Pääperiaate on, että \textbf{sanat, kaaviot ja
koodi} ovat kolme tapaa kommunikoida tietomallista kehittäjien ja
liiketoimintaihmisten välillä.

Tietomallin toteuttaminen olemassaolevassa ohjelmistossa oli rajattu jo
alunperinkin tämän projektin ulkopuolelle, ja se vaatisi vielä lisätyötä
ja suunnittelua. Syntynyttä GraphQL-rajapintaskeemaa olisi mahdollista
käyttää jatkotyön pohjana, ja näin parantaa myös vanhan ohjelmiston
sisäistä logiikkaa.

GraphQL-kieli on pohjimmiltaan melko yksinkertainen, mutta joitain sen
ominaisuuksia jäi tässä kartoittamatta. Esimerkiksi kielen tarjoamat
Input Typet, jotka mahdollistavat tallennettavan datan esittämisen
olioverkkona rajapintakyselyssä, sekä Union Typet, jotka tarjoavat tuen
polymorfismille, jäivät tässä vaiheessa kartoittamatta.
Jatkokysymykseksi siis jää, miten nämä monimutkaisemmat ominaisuudet
niveltyvät yhteen \glsdisp{ddd}{sovellusaluevetoisen suunnittelun}
kanssa.

Ehkä keskeisin johtopäätös insinöörityöstä on kuitenkin alalla laajasti
ja eri muodoissa toistettu näkemys. Olipa esittäjänä Osmo A. Wiio
(``viestintä epäonnistuu, paitsi sattumalta''), Gerald Weinberg (``No
matter how it looks at first, it's always a people problem.'') tai
Melvin Conway (``Organizations, who design systems, are constrained to
produce designs which are copies of the communication structures of
these organizations.''), lopulta ajatus kiertyy samaan johtopäätökseen:
ohjelmistossa esiintyvät ongelmat eivät useinkaan johdu teknologiasta
vaan ongelmista ihmisten välisessä kommunikaatiossa.
