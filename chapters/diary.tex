\hypertarget{viikko-1}{%
\subsection{Viikko 1}\label{viikko-1}}

Pidin palaverin Lauran kanssa, ja Pasin + Lauran kanssa. Sovimme
aihealueeksi Diariumin laskutuksen.

Lauran kanssa yhdessä alettiin keskustella laskutuksesta. Laura näytti,
miten Diariumissa laskuttamisen logiikka toimii, ja piirsin
tussitaululle esimerkkejä, miten logiikka voisi toimia.

Käytin yksinkertaista notaatiota, jossa merkitsin asioita laatikoilla,
ja niiden välisiä yksi moneen -suhteita.

Sain lopulta aikaiseksi mallin, josta kuva ohessa. {[}kuva mallista{]}
Tämä päätettiin toteuttaa.

Minulla meni torstaisen kokouksen jälkeen perjantai ja maanantai
perusrakenteen pystyttämiseen: Falcon, Ariadne GraphQL, Gunicorn ja
muutamia muita työkaluja. Lisäksi hommasin Pytest-testikirjaston, ja
opiskelin sen.

Lisäksi piti asentaa Apollo GraphQL-client ja Vue-cli, sekä plugin
Vue-Apollo.

Tiistaina sain ensimmäisen toiminnon, käyntien lisäämisen, valmiiksi.

Oli mielenkiintoista huomata, miten piirtämäni kuva ja tässä vaiheessa
aikaansaamani GraphQL-skeema muistuttivat läheisesti toisiaan.
GraphQL-queryjen sisältämät oliot rakensivat saman rakenteen, joka
tussitaululle piirsin. Sen sijaan GraphQL-mutaatioiden rooli ei ole
vielä auennut.

\hypertarget{viikko-2}{%
\subsection{Viikko 2}\label{viikko-2}}

Ensimmäisellä kehitykseen käyttämälläni viikolla (to-to) sain siis vain
ekan ominaisuuden valmiiksi. Sen jälkeen loppuviikosta tein vielä
käyntien laskutuksen. Lienee rehellistä arvioida, että noin viikon
kehitystyön myötä kaksi ``käyttäjätarinaa'' valmistui.

Perjantaina myös kirjoitin frontendia Vue-Apollolla ja käytin melkoisen
määrän aikaa kirjaston ominaisuuksien hahmottamiseen. Se ei kuulu
suoraan insinöörityön sisältöön, mutta toisaalta kuitenkin tarjoaa hyvän
kehyksen asioiden tekemiseenn.

\hypertarget{viikko-3}{%
\subsection{Viikko 3}\label{viikko-3}}

Sain ensimmäisen version laskuhärvelistä valmiiksi. Tavoite oli Lauran
kanssa pitää asiasta palaveri perjantaina, mutta se peruuntui, koska
Laura oli tulossa kipeäksi.

Laskuhärveli toimii sinänsä, ja on hankala miettiä, onko siitä apua.

Kuitenkin domain-tason konseptien mallintaminen GraphQL-skeemaksi toimii
hyvin. Esimerkki:

\begin{verbatim}
Type Invoice {
  numnber: Int
  sum: Float
  date: Date
}

type ConsolidatedInvoice {
  number: Int
  invoices: [Invoice]
}
\end{verbatim}
