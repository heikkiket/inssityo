\hypertarget{luxe4htuxf6kohdat-ja-tavoitteet}{%
\chapter{Lähtökohdat ja
tavoitteet}\label{luxe4htuxf6kohdat-ja-tavoitteet}}

\hypertarget{nordhealth-oy-lyhyesti}{%
\section{Nordhealth Oy lyhyesti}\label{nordhealth-oy-lyhyesti}}

Nordhealth Oy on vuonna 2001 perustettu ohjelmistopalveluyritys, joka
tekee toiminnanohjausjärjestelmiä kahdelle eri toimialalle: Provet Cloud
-järjestelmää eläinlääkäriklinikoille ja Diarium-järjestelmää
terapeuteille. Molemmat järjestelmät ovat web-pohjaisia sovelluksia.
Niitä käytetään kirjautumalla web-selaimen kautta. \textbf{(Tästä
puuttuu perustietoja, kuten työntekijöiden määrä, liikevaihto tms.)}

Nordhealthia voisi luonnehtia tyypilliseksi ohjelmistoalan yritykseksi:
se tekee erityisalalle suunnattua toiminnanohjausjärjestelmää.
Tyypillinen on myös järjestelmien kehityskaari: alunperin ne ovat olleet
työpöytäsovelluksia, ja siitä Nordhealth on muuntanut ne LAMP-alustalla
toimiviksi web-sovelluksiksi. Kun Web on kehittynyt, on otettu suunnaksi
sovellusten siirtäminen julkiseen pilveen, ja käyttöliittymän
rakentaminen erilliseksi yhden sivun JavaScript-sovellukseksi.

Diarium on Nordhealthin kahdesta järjestelmästä vanhempi. Se on
suunnattu terapeuteille: fysio-, toiminta-, puhe- ja psykoterapeuteille.
Järjestelmä on laajentunut yksinkertaisesta
potilaskortistojärjestelmästä suurenkin terapia-alan yrityksen tarpeita
vastaavaksi toiminnanohjausjärjestelmäksi, ja se on lisäksi myös
Valviran tarkoittama A-luokan potilastietojärjestelmä. Diariumia
käyttävä terapeutti voi siis lähettää käyntikirjaukset potilastiedon
sähköiseen Kanta-rekisteriin.

\hypertarget{motivointikappale}{%
\section{Motivointikappale}\label{motivointikappale}}

Tähän motivointikappale: miksi Diariumin tietomalli tarvitsee
parantelua, ja miksi juuri rajapinta on valikoitunut kohteeksi

\hypertarget{insinuxf6uxf6rityuxf6n-tavoite}{%
\section{Insinöörityön tavoite}\label{insinuxf6uxf6rityuxf6n-tavoite}}

Työn tavoitteena on selvittää, onko ohjelmiston tietomallia mahdollista
parannella rakentamalla GraphQL-rajapinta.

Lisäksi työn myötä on tavoitteena löytää parempi tietomalli osaan
Diarium-sovellusta.

Kolmas tärkeä tavoite on kehittää työmenetelmä, jonka avulla tietomallia
on mahdollista korjata.

Projektin edetessä rakennan pienen GraphQL-rajapinnan ja sitä
hyödyntävän prototyyppisovelluksen. Tämä toimii kokeilukenttänä, jonka
kautta työmenetelmää ja tietomallia etsitään.
