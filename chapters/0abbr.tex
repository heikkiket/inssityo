% Generate the glossary
\makeglossaries

% Acronyms, abbreviations, etc.

\newacronym{lamp}{LAMP}{LAMP (Linux, Apache, MySQL, PHP)}
\newacronym{orm}{ORM}{Object-relational mapper, oliot tietokantatauluiksi muuntava teknologia}
\newacronym{sql}{SQL}{Structured Query Language}
\newacronym{io}{I/O}{Input/Output}
\newacronym{ram}{RAM}{Random Access Memory}
\newacronym{php}{PHP}{Hypertext Preprocessor}
\newacronym{wysiwym}{WYSIWYM}{What You See Is What You Mean}
\newacronym{isbn}{ISBN}{International Standard Book Number}
\newacronym{url}{URL}{Uniform Resource Locator}
\newacronym{doi}{DOI}{Digital Object Identifier}


\newacronym{html}{HTML}{HyperText Markup Language}


% Glossary entries
\newglossaryentry{ddd}{
  name={Sovellusaluevetoinen suunnittelu},
  text={sovellusaluevetoinen suunnittelu},
  description={Menetelmä, jossa ohjelmistoa kehitetään läheisessä yhteistyössä sovellusalueen asiantuntijoiden kanssa. (myös: Liiketoimintavetoinen suunnittelu.) Engl. Domain Driven Design}
  }
\newglossaryentry{domain}{
  name={Sovellusalue},
  text={sovellusalue},
  description={Erityisala, jota sovellus käsittelee. Esimerkiksi pankkitoiminta tai vähittäiskauppa. myös: liiketoimintataso}
}

\newglossaryentry{crunching}{
  name={Tiedon rouhinta},
  text={tiedon rouhinta},
  description={Prosessi, jossa tunnistetaan sovellusalan erityiskäsitteitä ohjelmoijan ja sovellusalueen asiantuntijan välisen kommunikaation avulla. Engl. Knowledge crunghing}
  }
\newglossaryentry{dsl}{
  name={Täsmäkieli},
  text={täsmäkieli},
  description={Erityisalaan liittyvä formaali kieli. Esimerkiksi ohjelmointi- tai kuvauskieli. Engl. Domain-Specific language}
  }
\newglossaryentry{ubilang}{
  name={Kaikenkattava kieli},
  text={kaikenkattava kieli},
  description={Kokoelma käsitteitä ja sanastoa, joiden avulla ohjelmoijat ja sovellusalueen asiantuntijat keskustelevat kehitettävästä ohjelmistosta. Muotoutuu tiedon rouhintaprosessin tuloksena. Engl. Ubiquitous language}
  }
\newglossaryentry{domainmodel}{
  name={Sovellusaluemalli},
  text={sovellusaluemalli},
  description={Käsitteistä ja niiden välisistä suhteista koostuva, ohjelmakoodin muotoon kaapattava malli käsiteltävästä sovellusalueesta. Engl. Domain Model}
  }
\newglossaryentry{domainlayer}{
	name={Liiketoimintalogiikan taso},
	text={liiketoimintalogiikan taso},
	description={Ohjelmiston sisäinen osa, joka pitää sisällään liiketoimintaan liittyvän ohjelmalogiikan. Engl. Domain layer},
}
\newglossaryentry{entity}{
	name={Yksilötyyppi},
	text={yksilötyyppi},
	description={Ohjelmassa esiintyvä olio, jolla on identiteetti. Esimerkiksi yksittäinen asiakas tai lasku. Engl. Entity},
}
\newglossaryentry{deepermodel}{
  name={Syvempi malli},
  text={syvempi malli},
  description={Sovellusaluevetoisen suunnittelun myötä parantunut sovellusaluemalli. Syvempi malli tavoittaa aiempaa paremmin sovellusalan perimmäisen luonteen. Engl. Deeper Model}
}
\newglossaryentry{hakurakenne}{
  name={Assosiaatiotaulu},
  text={assosiaatiotaulu},
  description={Engl. Map. suom. myös {\em hakurakenne\/} ja {\em sanakirja\/}. Tietorakenne, joka sisältää joukon avain-arvo -pareja}
}

\newglossaryentry{rest}{
  name={REST},
  text={REST},
  description={Representational State Transfer. Arkkitehtoninen tyyli rajapintojen määrittelemiseen etenkin Web-sovelluksissa}
}

\newglossaryentry{graphql}{
  name={GraphQL},
  text={GraphQL},
  description={Facebookin kirjoittama kyselykieli, joka esittää manipuloitavan datan olioverkkona}
}

\newglossaryentry{http}{
  name={HTTP},
  text={HTTP},
  description={Hypertext Transfer Protocol. Webin perusprotokolla, jota käytetään hypertekstidokumenttien siirtoon}
}

\newglossaryentry{verkko}{
  name={Verkko},
  text={verkko},
  description={Tietorakenne, joka koostuu solmuista ja kaarista}
}

\newglossaryentry{domainexpert}{
  name={Alan asiantuntija},
  text={alan asiantuntija},
  description={Sovellusalueen esimerkiksi työnsä puolesta hyvin tunteva henkilö}
}

\newglossaryentry{kulkusuunta}{
  name={Kulkusuunta},
  text={kulkusuunta},
  description={Olioita yhdistävän kytköksen suunta}
}

\newglossaryentry{puhdasfunktio}{
  name={Puhdas funktio},
  text={puhdas funktio},
  description={Funktio, jonka palauttama tulos riippuu vain sen saamien parametrien arvoista. Puhdas funktio ei aiheuta sivuvaikutuksia esimerkiksi tulostamalla ruudulle tai tallentamalla tiedostoon}
}

\newglossaryentry{tyyppi}{
  name={Tyyppi},
  text={tyyppi},
  description={Lausekkeeseen sidottu määritys lausekkeen palauttaman arvon tietotyypistä. Esimerkiksi kokonaisluku tai merkkijono}
}

\newglossaryentry{aggregate}{
  name=Aggregaatti,
  text=aggregaatti,
  description={Useiden olioiden kokoelma, jolla on yksi juuriolio. Käsitellään ohjelmalogiikassa yhtenä kokonaisuutena.}
}
\newglossaryentry{repository}{
  name=Repositorio,
  text=repositorio,
  description=Rajapinta tiedon tallennusvälineeseen
}
\newglossaryentry{factory}{
  name=Tehdas-olio,
  text=tehdas-olio,
  description={Olio, joka vastaa toisten olioiden luomisesta}
}

