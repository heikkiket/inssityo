% Generate the glossary
\makeglossaries

% Acronyms, abbreviations, etc.

\newacronym{html}{HTML}{HyperText Markup Language}
\newacronym{sql}{SQL}{Structured Query Language}
\newacronym{io}{I/O}{Input/Output}
\newacronym{ram}{RAM}{Random Access Memory}
\newacronym{php}{PHP}{Hypertext Preprocessor}
\newacronym{wysiwym}{WYSIWYM}{What You See Is What You Mean}
\newacronym{isbn}{ISBN}{International Standard Book Number}
\newacronym{url}{URL}{Uniform Resource Locator}
\newacronym{doi}{DOI}{Digital Object Identifier}

% Glossary entries
\newglossaryentry{ddd}{
  name={Sovellusaluevetoinen suunnittelu},
  description={engl. Domain Driven Design, myös Liiketoimintavetoinen suunnittelu}
  }
\newglossaryentry{crunching}{
  name={tiedon rouhinta},
  description={engl. Knowledge crunghing}
  }
\newglossaryentry{ubilang}{
  name={kaikenkattava kieli},
  description={engl. Ubiquitous language}
  }
\newglossaryentry{domainmodel}{
  name={sovellusaluemalli},
  description={engl. Domain Model}
  }
\newglossaryentry{domainlayer}{
	name={liiketoimintalogiikan taso},
	description={engl. Domain layer},
}
\newglossaryentry{entity}{
	name={yksilötyyppi},
	description={engl. Entity},
}
\newglossaryentry{deepermodel}{
  name={Syvempi malli},
  description={engl. Deeper Model}
}
\newglossaryentry{hakurakenne}{
  name={Hakurakenne},
  description={engl. Map. suom. myös {\em assosiaatiotaulu\/} ja {\em sanakirja\/}. Tietorakenne, joka sisältää joukon avain-arvo -pareja.}
}
