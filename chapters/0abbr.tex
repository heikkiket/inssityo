% Generate the glossary
\makeglossaries

% Acronyms, abbreviations, etc.

\newacronym{html}{HTML}{HyperText Markup Language}
\newacronym{sql}{SQL}{Structured Query Language}
\newacronym{io}{I/O}{Input/Output}
\newacronym{ram}{RAM}{Random Access Memory}
\newacronym{php}{PHP}{Hypertext Preprocessor}
\newacronym{wysiwym}{WYSIWYM}{What You See Is What You Mean}
\newacronym{isbn}{ISBN}{International Standard Book Number}
\newacronym{url}{URL}{Uniform Resource Locator}
\newacronym{doi}{DOI}{Digital Object Identifier}

% Glossary entries
\newglossaryentry{ddd}{
  name={Sovellusaluevetoinen suunnittelu},
  description={engl. Domain Driven Design, myös Liiketoimintavetoinen suunnittelu}
  }
\newglossaryentry{domain}{
  name={Sovellusalue},
  description={Erityisala, jota sovellus käsittelee. Esimerkiksi pankkitoiminta tai vähittäiskauppa. myös: liiketoimintataso}
}

\newglossaryentry{crunching}{
  name={tiedon rouhinta},
  description={engl. Knowledge crunghing}
  }
\newglossaryentry{dsl}{
  name={täsmäkieli},
  description={engl. Domain-Specific language}
  }
\newglossaryentry{ubilang}{
  name={kaikenkattava kieli},
  description={engl. Ubiquitous language}
  }
\newglossaryentry{domainmodel}{
  name={sovellusaluemalli},
  description={engl. Domain Model}
  }
\newglossaryentry{domainlayer}{
	name={liiketoimintalogiikan taso},
	description={engl. Domain layer},
}
\newglossaryentry{entity}{
	name={yksilötyyppi},
	description={engl. Entity},
}
\newglossaryentry{deepermodel}{
  name={Syvempi malli},
  description={engl. Deeper Model}
}
\newglossaryentry{hakurakenne}{
  name={Hakurakenne},
  description={engl. Map. suom. myös {\em assosiaatiotaulu\/} ja {\em sanakirja\/}. Tietorakenne, joka sisältää joukon avain-arvo -pareja.}
}

\newglossaryentry{rest}{
  name={REST},
  description={Representational State Transfer. Arkkitehtoninen tyyli rajapintojen määrittelemiseen etenkin Web-sovelluksissa.}
}

\newglossaryentry{graphql}{
  name={GraphQL},
  description={Facebookin kirjoittama kyselykieli, joka esittää manipuloitavan datan olioverkkona.}
}

\newglossaryentry{http}{
  name={HTTP},
  description={Hypertext Transfer Protocol. Webin perusprotokolla, jota käytetään hypertekstidokumenttien siirtoon.}
}

\newglossaryentry{verkko}{
  name={Verkko},
  description={Tietorakenne, joka koostuu solmuista ja kaarista.}
}

\newglossaryentry{domainexpert}{
  name={alan asiantuntija},
  description={Sovellusalueen esimerkiksi työnsä puolesta hyvin tunteva henkilö}
}

\newglossaryentry{deepmodel}{
  name={syvä malli},
  description={Sovellusalueen esimerkiksi työnsä puolesta hyvin tunteva henkilö}
}

\newglossaryentry{kulkusuunta}{
  name={kulkusuunta},
  description={olioita yhdistävän kytköksen suunta}
}

\newglossaryentry{puhdasfunktio}{
  name={puhdas funktio},
  description={Funktio, jonka palauttama tulos riippuu vain sen saamien parametrien arvoista. Puhdas funktio ei aiheuta sivuvaikutuksia esimerkiksi tulostamalla ruudulle tai tallentamalla tiedostoon.}
}

\newglossaryentry{tyyppi}{
  name={tyyppi},
  description={lausekkeeseen sidottu määritys lausekkeen palauttaman arvon tietotyypistä. Esimerkiksi kokonaisluku tai merkkijono.}
}

\newglossaryentry{aggregate}{name=aggregaatti, description=useiden olioiden kokoelma}