\documentclass[12pt,a4paper,oneside,article]{memoir}%Do not touch this first line ;)

% Global information (title of your thesis, your name, degree programme, major, etc.)

\def\bilingual{yes}%For Finnish students, you must have 2 abstracts, one in English and one in your native language (Finnish or Swedish), so "yes", your thesis is bilingual.
%\def\bilingual{no}%For international student writing in English, only one language and one abstract.

\def\thesislang{finnish} %change this depending on the main language of the thesis.
%\def\thesislang{english} % "english" is the only other supported language currently. If someone has the swedish, please contribute!

\def\secondlang{english} %if the main language is Finnish (or Swedish), you must have 2 abstracts (one in Finnish (or Swedish) and one in English)
%\def\secondlang{finnish}
%If the main language is English and that you are native Finnish (or Swedish) speaker, you must have also abstract in your native language on top of the English one.

\author{Heikki Ketoharju} %your first name and last name

%\def\alaotsikko{Alaotsikko/Subtitle} %DISABLED, seems not to be an option with the 2018 template. If you really need it, uncomment and modify style/title.tex accordingly.

%License
%When publishing your thesis to theseus.fi, you can keep all rights reserved to you,
%or use one of the Creative Commons https://creativecommons.org/licenses/?lang=en
%This attribute will set the license in the metadata of the generated pdf.
%possible options (case sensitive):
%all (keep all rights reserved to yourself)
%by (Attribution)
%by-sa (Attribution-ShareAlike)
%by-nd (Attribution-NoDerivs)
%by-nc (Attribution-NonCommercial)
%by-nc-sa (Attribution-NonCommercial-ShareAlike)
%by-nc-nd (Attribution-NonCommercial-NoDerivs)
%Note that theseus do not accept dedication to public domain CC0
\def\thesiscopy{by-sa}

%Finnish section, for title/abstract
\def\otsikko{Jaettua kieltä etsimässä – GraphQL-rajapinta tietomallin kohentelun
välineenä}
\def\tutkinto{Insinööri (AMK)} % change to your needs, e.g. "YAMK", etc.
\def\kohjelma{Tieto– ja viestintätekniikka}
\def\suuntautumis{Ohjelmistotuotanto}
\def\thesisfi{Insinöörityö}%was Opinnäytetyö
\def\ohjaajat{
Lehtori Vesa Ollikainen\newline
Development Lead Pasi Nissinen
}
\def\tiivistelma{
Insinöörityössä pyrittiin kohentamaan ohjelmiston tietomallia kehittämällä
GraphQL-rajapinta sovellusaluevetoisen suunnittelun keinoin. Tavoitteena oli
parantaa ohjelmiston tietomallia ja luoda työprosessi tietomallin
parantelemiseen.\newline

Työn kuluessa arvioitiin GraphQL-rajapinnan soveltuvuutta tietomallin
parantamisen välineeksi ja tutkittiin, kuinka hyvin teknologia soveltuu
sovellusaluevetoisen suunnittelun välineeksi. Tavoitteena oli hahmottaa
GraphQL:n rakenteen ja sovellusaluevetoisen suunnittelun yhtymäkohtia.\newline

Insinöörityössä laadittiin pieni prototyyppisovellus, jonka kautta tietomallin
parantamisen prosessia kehitettiin ja GraphQL-teknologiaa testattiin. Tämä
prototyyppi toimi testialustana sovellusaluevetoisen suunnittelun käytäntöjen ja
sovellusalan käsitteiden ymmärtämiseen.\newline

GraphQL soveltuu hyvin sovellusaluevetoisen suunnittelun työkaluksi. Sen tapa
esittää rajapinnan takana oleva järjestelmä olioiden verkkona ja sen
ohjelmointikielistä riippumaton täsmäkielimuotoinen toteutus tekevät siitä hyvän
teknologian vanhan järjestelmän tietomallin kohenteluun.\newline

Insinöörityön tuloksena syntyi Notkean tietomallin paranteluksi nimetty
työmalli, jonka avulla tietomallia voi selkeyttää. Pääperiaate on, että sanat,
kaaviot ja koodi ovat kolme tapaa kommunikoida tietomallin sisältämiä ideoita
kehittäjien ja sovellusalan asiantuntijoiden välillä.\newline

Työn tuloksena syntyneen työmallin avulla on mahdollista rakentaa iäkkään
ohjelmiston tietorakennetta parantava GraphQL-rajapinta.
}
\def\avainsanat{GraphQL, sovellusaluevetoinen suunnittelu, rajapinta,
  olioverkko, extreme programming}
\def\aihe{Insinöörityössä etsittiin keinoja parantaa iäkkään sovelluksen tietomallia kehittämällä GraphQL-rajapinta sovellusaluevetoisen suunnittelun keinoin. Tuloksena syntyi työmalli, jolla tietomallia voidaan parantaa.}%for the pdf metadata/properties. If not used, empty it and also the \def\subject.

%English section, for title/abstract
\title{Your title here}
\def\metropoliadegree{Bachelor of Engineering} % change to your needs, e.g. "master", etc.
\def\metropoliadegreeprogramme{Information Technology}
\def\metropoliaspecialisation{Software Engineering}
\def\thesisen{Bachelor’s Thesis} % change to your need, e.g. master's
\def\metropoliainstructors{
Vesa Ollikainen, Senior Lecturer\newline
Pasi Nissinen, Development Lead
}
\def\abstract{
In this Bachelors' the aim was to enchange the data model of a software by
developing a GraphQL interface with means of Domain Driven Design. The goal was
both to enchange data model and to create a process for enchanging the data
model.\newline

During the work GraphQL was evaluated as a means of enchangement for the data
model. It was researched how well this technology suits as a tool for domain
driven development. The goal was to understand confluences of Domain Driven
Design and a structure of GraphQL API.\newline

In this Bachelors' a small prototype software was developed where the nature of
GraphQL was tested and the process for enchanging the data model was developed.
This prototype was an environment for understanding practises of Domain Driven
Design and the consepts of a domain.\newline

GraphQL suits well as a tool for Domain Driven Design. The way how it presents
the system behind the API as an object graph and its programming language
independent implementation in a form of domain specific language make it as a
good technology for enchanging the data model of legacy system.\newline

As a result of the work, a working model for clarifying the data model was
created, called \em{Notkea tietomallin parantelu}. The primary priciple is that
words, diagrams and code are three ways to communicate about ideas included in data
model, between developers and domain experts.\newline

With the resulting working model it is possible to build a GraphQL API that
enchanhes ti data model of a legacy system.
}
\def\metropoliakeywords{GraphQL, Domain Driven Design, API,
  object graph, extreme programming}
\def\subject{short description of the thesis. Max 255 characters.}%for the pdf metadata/properties. If not used, empty it and also the \def\aihe.
