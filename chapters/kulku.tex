\vspace{21.5pt}

\hypertarget{tyuxf6n-kulku}{%
\section{Työn kulku}\label{tyuxf6n-kulku}}

Työn aluksi valitsimme yhdessä ohjaajani sekä tiimin tuoteomistaja
Lauran kanssa aiheen työhön. Laskutusta koskevat osat ohjelmistossa ovat
suhteellisen monimutkaisia, ja sisältävät käsitteellisiä epäselvyyksiä.
Valitsimme laskutuksen sisältä tapauksen, jossa hoitokäynti tulee voida
jakaa usealle eri maksajalle osoitetuille laskuille, ja nämä laskut
tulee voida hyvittää itsenäisesti.

Tämän jälkeen ryhdyin pitämään tiimin tuoteomistaja Lauran kanssa
kokokouksia, joissa suunnittelimme mallin rakennetta. Pyrin noudattamaan
työskentelyssä Eric Evansin esittämää tiedon rouhimisen periaatetta,
jossa suunnittelu ja ohjelmistokehitys limittyvät keskenään.

Pidimme lyhyitä suunnittelukokouksia, joissa pohdimme, millainen mallin
tulisi olla. Kokousten välillä kirjoitin ohjelmiston, joka vastasi
pohdintojamme. Seuraavassa kokouksessa katsoimme, miten ohjelma toimii,
ja lisäsimme malliin uusia piirteitä.

Käytin apuvälineenä tussitaulua, johon piirsin erilaisia ehdotuksia
malleiksi. Oleellista on, että taulun pystyi pyyhkimään nopeasti, ja
piirtämään uuden mallin. Otimme myös valokuvia mallinnussession eri
vaiheista. En kuitenkaan halunnut, että piirretyt mallit sanelevat
ohjelmiston rakennetta. Tärkein mittari on ohjelmakoodi, ja sen
ilmaisema rakenne. Piirrokset toimivat apuna, tämän rakenteen kuvaajina.

Ensimmäisesssä kokouksessa hahmottelimme yksinkertaisen mallin, jossa
käynnit liitetään laskuihin, laskut koontilaskuihin ja koontilaskut
hyvityslaskuihin. Toteutin tämän mallin parin viikon kuluessa, jonka
jälkeen pidimme uuden tapaamisen.

\begin{verbatim}
+--------------+       +-----------+       +---------------------+
|  Appointment | ----> |  Invoice  | ----> | ConsolidatedInvoice |
+--------------+       +-----------+       +---------------------+
                                                |     +---------------+ 
                                                ----> | ServiceCredit | 
                                                      +---------------+ 
\end{verbatim}

Tavoitteenani oli jokaisen tapaamisen myötä rakentaa hieman
monipuolisempi ja paremmin ohjelmiston käyttäjien tarpeita vastaava
malli. Toisinaan tällaiset laajentamispyrkimykset voivat myös johtaa
äkilliseen läpimurtoon, jonka myötä syntyy
\gls{deepermodel}.\cite{evans:ddd}

Tapaamisissa kävin joka kerta läpi, mitkä käsitteelliset asiat olivat
ohjelmoidessa vaivanneet. Esimerkiksi toisella tapaamiskerralla esitin
suurimmaksi ongelmaksi sen, että käynnin ja laskun välillä on suora
kytkös. Ohjelmoidessa tämä kytkös tuli koko ajan huomioida, ja varoa
aiheuttamasta ongelmia. Pyysin Lauraa kertomaan enemmän siitä, mitä
käynnin laskuttaminen oikeastaan tarkoittaa, ja hän piti lyhyen
yhteenvedon laskuttamisen periaatteista. Huomioni kiinnittyi puheessa
esiintyneeseen termiin \textbf{Laskutusperuste}. Tämä tuntui valtavan
kiinnostavalta, ja lähdimme tarkastelemaan sitä eri puolilta.

Tapaamisen jälkeisen viikon kehitystyötä ohjasi nyt uusi ajattelutapa:
käyntiä sinänsä ei liitetä laskuun, vaan käynti laskutetaan, mikäli
laskutusperuste täyttyy. Tämän tuloksena syntyi melko yksinkertainen
malli:

\begin{verbatim}
+--------------+       +--------------+       +----------------+
|  Appointment | ----> |  ServiceRow  | ----> |  ServiceCredit |
+--------------+       +--------------+       +----------------+
\end{verbatim}

\hypertarget{ubiquitous-language-kuxe4ytuxe4nnuxf6ssuxe4}{%
\subsection{Ubiquitous Language
käytännössä}\label{ubiquitous-language-kuxe4ytuxe4nnuxf6ssuxe4}}

Kuuluisivatkohan nämä oikeastaan tulosten tarkastelun tai yhteenvedon
alle?

Pitäessäni suunnittelukokouksia yhdessä tuoteomistajan kanssa, pyrin
koko ajan kuuntelemaan tarkalla korvalla, minkälaisia sanoja käytimme.
Tällä tavoin onnistuin nappaamaan joitain tärkeitä käsitteitä, joita
pystyi käyttämään mallin pohjana. Toisen kokouksemme aikana esiin
noussut \textbf{Laskutusperuste} oli juuri tällainen käsite.

Eric Evans mainitsee, että \textbf{Kaikenkattavan kielen} rakentamisessa
oleellista on löytää sanat, joita alan asiantuntijat käyttävät.

\hypertarget{kuxe4sitekarttojen-ja-graphql-skeeman-yhteys}{%
\subsection{Käsitekarttojen ja GraphQL-skeeman
yhteys}\label{kuxe4sitekarttojen-ja-graphql-skeeman-yhteys}}

Olin yllättynyt, miten täsmällisesti piirtämäni käsitekartat oli
mahdollista ilmaista GraphQL-skeeman avulla. Olioiden suhteet siirtyivät
vaivattomasti skeeman sisälle hierarkioiksi.
