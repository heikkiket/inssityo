\vspace{21.5pt}

\hypertarget{tyuxf6n-kulku}{%
\section{Työn kulku}\label{tyuxf6n-kulku}}

Tähän tulee kuvaus työn kulusta päiväkirjan pohjalta.

Kuuluisivatkohan nämä oikeastaan tulosten tarkastelun tai yhteenvedon
alle?

\hypertarget{ubiquitous-language-kuxe4ytuxe4nnuxf6ssuxe4}{%
\subsection{Ubiquitous Language
käytännössä}\label{ubiquitous-language-kuxe4ytuxe4nnuxf6ssuxe4}}

Pitäessäni suunnittelukokouksia yhdessä tuoteomistajan kanssa, pyrin
koko ajan kuuntelemaan tarkalla korvalla, minkälaisia sanoja käytimme.
Tällä tavoin onnistuin nappaamaan joitain tärkeitä käsitteitä, joita
pystyi käyttämään mallin pohjana. Toisen kokouksemme aikana esiin
noussut \textbf{Laskutusperuste} oli juuri tällainen käsite.

Eric Evans mainitsee, että \textbf{Kaikenkattavan kielen} rakentamisessa
oleellista on löytää sanat, joita alan asiantuntijat käyttävät.

\hypertarget{kuxe4sitekarttojen-ja-graphql-skeeman-yhteys}{%
\subsection{Käsitekarttojen ja GraphQL-skeeman
yhteys}\label{kuxe4sitekarttojen-ja-graphql-skeeman-yhteys}}

Olin yllättynyt, miten täsmällisesti piirtämäni käsitekartat oli
mahdollista ilmaista GraphQL-skeeman avulla. Olioiden suhteet siirtyivät
vaivattomasti skeeman sisälle hierarkioiksi.
