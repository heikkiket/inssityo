\hypertarget{johdanto}{%
\chapter{Johdanto}\label{johdanto}}

Pitkäikäinen, jatkuvasti kehitetty ohjelma mutkistuu herkästi. Jos
kehitystiimi ei pidä varaansa, alun perin yksinkertaisesta ja selkeästä
ohjelmakoodista ja suoraviivaisesta tietorakenteesta kasvaa hankalasti
hallittava kokonaisuus.

Sotkuiseenkin ohjelmistoon voi kuitenkin saada selvyyttä taitavalla
kohentelulla ja kehittämisellä. Kun toimintoja alkaa jakaa osiin, ja
rakentaa osien välille selkeitä nivelkohtia, avautuu mahdollisuus tehdä
hedelmällisiä parannuksia.

Web-sovelluksessa eräs merkittävimpiä nivelkohtia on sovelluksen
käyttöliittymän ja logiikkaosan välinen HTTP-rajapinta. Tässä
insinöörityössä tarkastellaan, voiko tämän merkittävän nivelkohdan
uudistamisen kanssa yhtä aikaa myös kohentaa ohjelmiston sisäistä
logiikkaa.

Työssä perehdytään GraphQL-rajapinnan toimintaan Domain Driven Designin
näkökulmasta. Tuloksena syntyy toimintamalli, jonka avulla tietomallia
voidaan kohentaa.
