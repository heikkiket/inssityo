% Introduction

\chapter{Johdanto}

Pitkäikäinen, jatkuvasti kehitetty ohjelma mutkistuu herkästi. Jos kehitystiimi
ei pidä varaansa, alun perin yksinkertaisesta ja selkeästä ohjelmakoodista ja suoraviivaisesta tietorakenteesta kasvaa hankalasti hallittava kokonaisuus.

Sotkuiseenkin ohjelmistoon voi kuitenkin saada selvyyttä taitavalla kohentelulla
ja kehittämisellä. Kun toimintoja alkaa jakaa osiin, ja rakentaa osien välille
selkeitä nivelkohtia, avautuu mahdollisuus tehdä hedelmällisiä parannuksia.

Web-sovelluksessa eräs merkittävimpiä nivelkohtia on sovelluksen käyttöliittymän
ja logiikkaosan välinen HTTP-rajapinta. Tässä insinöörityössä tarkastellaan,
voiko tämän rajapinnan uudistamisen kanssa yhtä aikaa myös kohentaa ohjelmiston sisäistä logiikkaa.

Työ tehdään ohjelmistopalveluyritys Nordhealth Oy:n tilauksesta.
 
\section{Nordhealth Oy lyhyesti}


Nordhealth Oy on vuonna 2001 perustettu ohjelmistopalveluyritys, joka tekee toiminnanohjausjärjestelmiä kahdelle eri toimialalle: Provet Cloud -järjestelmää eläinlääkäriklinikoille ja Diarium-järjestelmää terapeuteille. Molemmat järjestelmät ovat web-pohjaisia sovelluksia. Niitä käytetään kirjautumalla web-selaimen kautta. (Tästä puuttuu perustietoja, kuten työntekijöiden määrä, liikevaihto tms.)

Nordhealthia voisi luonnehtia tyypilliseksi ohjelmistoalan yritykseksi: se tekee
erityisalalle suunnattua toiminnanohjausjärjestelmää. Alunperin järjestelmät
ovat olleet työpöytäsovelluksia, ja siitä Nordhealth on muuntanut ne
LAMP-alustalla toimiviksi web-sovelluksiksi. Kun Web on kehittynyt, on otettu
suunnaksi sovellusten siirtäminen julkiseen pilveen, ja käyttöliittymän
rakentaminen erilliseksi yhden sivun JavaScript-sovellukseksi.

Diarium on Nordhealthin kahdesta järjestelmästä vanhempi. Se on suunnattu terapeuteille: fysio-, toiminta-, puhe- ja psykoterapeuteille. Järjestelmä on laajentunut yksinkertaisesta potilaskortistojärjestelmästä suurenkin terapia-alan yrityksen tarpeita vastaavaksi toiminnanohjausjärjestelmäksi, ja se on lisäksi myös Valviran tarkoittama A-luokan potilastietojärjestelmä. Diariumia käyttävä terapeutti voi siis lähettää käyntikirjaukset potilastiedon sähköiseen Kanta-rekisteriin.

\section{Insinöörityön tavoite}

Työn tavoitteena on analysoida pientä osaa Diariumin tietomallista, ja etsiä
parempia tapoja sen toteuttamiseen. Diariumin HTTP-rajapintaa halutaan muista
syistä kehittää, ja tämä tarjoaa työni pääkysymyksen: onko ohjelmiston
tietomallia mahdollista kehittää rakentamalla GraphQL-rajapinta?

GraphQL on teknologiana uusi ja erilainen verrattuna vanhaan HTTP REST
-rajapintaan. Tarvitaan siis hyviä perusteita siihen siirtymiselle.

Taustateoksena työssä käytän Eric Evansin kirjaa Doman Driven Design.

Työn tuloksena en ajattele syntyvän valmista ja parempaa tietomallia, vaan
pikemminkin prosessi tietomallin kehittämiseksi rajapinnan uusimisen ohella.