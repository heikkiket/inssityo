% Introduction

\chapter{Introduction}

Pitkäikäinen, jatkuvasti kehitetty ohjelma mutkistuu herkästi. Jos kehitystiimi
ei pidä varaansa, alun perin yksinkertaisesta ja selkeästä ohjelmakoodista ja suoraviivaisesta tietorakenteesta kasvaa hankalasti hallittava kokonaisuus.

IT-ala myös kehittyy vauhdilla. Uusia teknologioita syntyy kaiken aikaa, ja kymmenen vuotta sitten ajantasaisilla teknologioilla käynnistynyt projekti on päivittämisen tarpeessa.

Tässä insinöörityössä tarkastellaan, voiko teknologioiden päivittämisen kanssa yhtä aikaa myös kohentaa ohjelmiston sisäistä logiikkaa.

Työ tehdään ohjelmistopalveluyritys Nordhealth Oy:n tilauksesta.
 
\section{Nordhealth Oy lyhyesti}


Nordhealth Oy on vuonna 2001 perustettu ohjelmistopalveluyritys, joka tekee toiminnanohjausjärjestelmiä kahdelle eri toimialalle: Provet Cloud -järjestelmää eläinlääkäriklinikoille ja Diarium-järjestelmää terapeuteille. Molemmat järjestelmät ovat web-pohjaisia sovelluksia. Niitä käytetään kirjautumalla web-selaimen kautta. (Tästä puuttuu perustietoja, kuten työntekijöiden määrä, liikevaihto tms.)

Diarium on näistä kahdesta järjestelmästä vanhempi. Se on suunnattu terapeuteille: fysio-, toiminta-, puhe- ja psykoterapeuteille. Järjestelmä on laajentunut yksinkertaisesta potilaskortistojärjestelmästä suurenkin terapia-alan yrityksen tarpeita vastaavaksi toiminnanohjausjärjestelmäksi, ja se on lisäksi myös Valviran tarkoittama A-luokan potilastietojärjestelmä. Diariumia käyttävä terapeutti voi siis lähettää käyntikirjaukset potilastiedon sähköiseen Kanta-rekisteriin.
